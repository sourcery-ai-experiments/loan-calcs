
\section{Definitions}\label{sec:definitions}


\subsection{Notation}\label{subsec:notation}

A loan is a fixed value of money borrowed by an entity and usually repaid over a series of instalments.\ The following notation is used for each of the components of the loan:
\begin{itemize}
    \item[\(L\):] Loan amount (decimal).
    \item[\(R\):] Periodic interest rate (decimal).
    \item[\(N\):] Total number of repayments (integer).
    \item[\(P\):] Total periodic repayment value (decimal).\ If this changes over time, the value at period \(n\) is denoted \(P_{n}\).
    \item[\(b\):] Whether the interest is applied before or after the repayment (boolean).
    \item[\(B_{n}\):] The balance on the loan at period \(n\) (decimal).\ Note that \(B_{0} = L\).
\end{itemize}

The periodic repayment for a loan usually has (at least) 2 components:
\begin{itemize}
    \item[\(P_{P, n}\):] The principal part of the periodic repayment value at period \(n\), which is paying off the original money that was borrowed.
    \item[\(P_{I, n}\):] The interest part of the periodic repayment value at period \(n\), which is paying off the interest applied on the loan.
\end{itemize}

In `real life', a loan can have other components such as fees.\ These are outside the scope of these calculations.


\subsection{Helpful Formulae}\label{subsec:helpful-formulae}

The finite sum of a geometric series:
\begin{equation*}
    \sum_{i = 0}^{n - 1} a^{i} = \frac{a_{n} - 1}{a - 1}
\end{equation*}

(See \url{https://en.wikipedia.org/wiki/Geometric\_series\#Finite\_series})


\subsection{Loan~Types}\label{subsec:loan-types}
There are 3 types of loans discussed here:
\begin{enumerate}
    \item Fixed Repayment Loans
    \item Fixed Principal Loans
    \item Interest Only Loans
\end{enumerate}
